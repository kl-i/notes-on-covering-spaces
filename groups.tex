\documentclass[./main.tex]{subfiles}
\begin{document}
  
\begin{prop}[Decomposition to Path-Connected Components]
  
  Let $B\in\TOP$ be locally path connected and semi-locally simply connected.
  Let $B = \bigsqcup_{B_i \in I} B_i$ where $I$ is the set of 
  path-connected components of $B$.
  Then \begin{enumerate}
    \item $\Pi_1 B \simeq \coprod_{B_i \in I} (\Pi_1 B_i)$.
    \item we have a commutative square 
    \begin{cd}
      \COV(B) \ar[d,"\simeq"{swap}] \ar[r,"\simeq"]
        & \Pi_1 B\SET \ar[d,"\simeq"]\\
      \prod_{B_i \in I} \COV(B_i) \ar[r,"\simeq"]
        & \prod_{B_i \in I} \Pi_1 B_i \SET
    \end{cd}
  \end{enumerate}
\end{prop}
\begin{proof}
  \textit{(1)} ok. 
  \textit{(2)} The square commutes. 
  By base change of covering space,
  we get the functor $\COV(B) \to \prod_{B_i \in I} \COV(B_i)$. 
  For full and faithful, note that every $X \in \COV(B)$ is 
  isomorphic to $\coprod_{B_i \in I} X_i$ where 
  $X_i$ is the preimage of $B_i$.
  Essentially surjective is okay and 
  this shows left vertical functor is equivalence. 
  The other vertical equivalence and the bottom horizontal equivalence 
  are formal. 

\end{proof}

\begin{rmk}
  The above justifies restricting to the case with 
  the additional assumption of $B$ path connected. 
\end{rmk}

\begin{prop}[Fundamental Theorem of Covering Spaces (Groups Version)]
  
  Let $B \in \TOP$ be path connected.
  For $b \in B$, 
  use $\pi_1(B,b)$ to refer to the 
  single object category consisting of $b$ and 
  automorphisms in $\Pi_1 B$ as the only morphisms.
  Then for any $b \in B$, 
  there's an equivalence of categories \[
    \pi_1(B,b) \overset{\simeq}{\to} \Pi_1 B
  \]
  which induces an equivalence of categories : 
  \[
    \Pi_1 B \SET \overset{\simeq}{\to} \pi_1(B,b)\SET
  \]

  Thus, when $B$ is path connected, locally path connected,
  semi-locally simply connected, 
  we have equivalence of categories :
  \begin{cd}
    \COV(B) \ar[r,"\FIB","\simeq"{swap}] \ar[dr,"\FIB(b)"{swap}]
      & \Pi_1B\SET \ar[d,"\simeq"] \\
      & \pi_1(B,b)\SET
  \end{cd}
  where 
  \begin{enumerate}
    \item the equivalence restricts to 
    an equivalence between path connected coverings of $B$ and 
    transitive $\pi_1(B,b)$-actions. 
    \item for $X \in \COV(B)$ path connected and any $x \in \FIB_X(b)$, 
    we have \[
      \AUT_{\COV(B)} X \iso 
      \AUT_{\pi_1(B,b)} \FIB_X(b) \iso 
      N\stab(x)/\stab(x)
    \]
    In particular, 
    $\AUT_{\COV(B)}(X)$ acts transitively on $\FIB_X(b)$ 
    if and only if $\stab(x)$ is normal in $\pi_1(B,b)$.
  \end{enumerate}
\end{prop}
\begin{proof}
  The forgetful functor $\pi_1(B,b) \to \Pi_1B$ is fully faithful
  and is essentially surjective because $B$ is path connected.
  $\Pi_1 B \SET \simeq \pi_1(B,b)\SET$ is formal.
  For $\FIB(b) : \COV(B) \to \pi_1(B,b)\SET$ being an equivalence,
  note that a composition of equivalences form an equivalence. 

  \textit{(1)}
  Now let $X \in \COV(B)$.
  $X$ is path connected if and only if $\FIB_X$ is a transitive 
  $\Pi_1B$-action. 
  This implies $\FIB_X(b)$ is a transitive $\pi_1(B,b)$-action. 
  Conversely, $B$ path connected implies 
  that any two fibers $\FIB_X(b), \FIB_X(b_1)$ biject by some 
  $\FIB_X[\ga]$ with $[\ga] \in \Pi_1B (b,b_1)$ so 
  $\FIB_X(b)$ being a transitive $\pi_1(B,b)$-action gives the desired result.
  
  \textit{(2)}
  Let $X \in\COV(B)$ be path connected and $x \in \FIB_X(b)$.
  $\AUT_{\COV(B)} X \iso \AUT_{\pi_1(B,b)} \FIB_X(b)$ comes from 
  the equivalence of categories. 
  For $\AUT_{\pi_1(B,b)} \FIB_X(b) \iso N\stab(x)/\stab(x)$
  and the in particular,
  this is a general group theoretic fact : 
  \begin{lem}[Automorphisms of Non-empty Transitive Actions]
    Let $G \in \GRP$ and $X \in G\SET$ be transitive and $x \in X$.
    Then 
    \begin{enumerate}
      \item $\AUT_{G\SET} X \iso N\stab(x)/\stab(x)$.
      \item $\AUT_{G\SET} X$ acts transitively on $X$ if and only if 
      $N\stab(x) = G$.
    \end{enumerate}
    \begin{proof1}
      \textit{(1)} Define : 
      \begin{align*}
        N\stab(x) \to \AUT_{G\SET} X,
        \si \mapsto (\si_0(x) \mapsto (\si_0 \circ \si)(x))
      \end{align*}
      The group morphism is well-defined 
      because $\stab(x)$ is normal in $N\stab(x)$.
      To show surjective, 
      let $\ph \in \AUT_{G\SET} X$.
      Since $X$ is a transitive $G$-set,
      $\ph$ is determined by what it does to $x$.
      There exists $\si \in G$ with $\si(x) = \ph(x)$.
      It suffices to show $\si \in N\stab(x)$. 
      This is true by a quick computation.
      It is clear that the kernel of the group morphism is 
      $\stab(x)$ so we have the desired group isomorphism.

      \textit{(2)} Assume $\AUT_{G\SET} X$ acts transitively on $X$.
      Let $\si \in G$. Then there exists $\si_0 \in N\stab(x)$
      such that $\si_0(x) = \si(x)$.
      It follows that $\si \in \si_0 \stab(x) \subs N\stab(x)$.
      The converse is clear. 
    \end{proof1}
  \end{lem}
\end{proof}

\begin{rmk}
  The following is good for computing fundamental groups.
\end{rmk}

\begin{prop}[Universal Cover]
  
  Let $B \in \TOP$ be path connected, locally path connected,
  semi-locally simply connected.
  Let $b \in B$ and consider $\pi_1(B,b)$ with 
  the obvious left action on itself. 
  Then for $X \in \COV(B)$, TFAE : 
  \begin{enumerate}
    \item (Algebra) 
    $\FIB_X(b)$ is a non-empty, transitive and free $\pi_1(B,b)$-action,
    in other words, $\FIB_X(b) \iso \pi_1(B,b)$ as $\pi_1(B,b)$-sets.
    In this case, we call $\FIB_X(b)$ a \emph{$\pi_1(B,b)$-torsor}.
    \item (Topology) $X$ is path connected and 
    there exists $x \in \FIB_X(b)$ with $\pi_1(X,x)$ is trivial. 
  \end{enumerate}
  We call $X$ \emph{universal} when it satisfies any 
  (and thus both) of the above.
  In this case, \[
    \AUT_{\COV(B)} X \iso \AUT_{\pi_1(B,b)} \pi_1(B,b)
    \iso \pi_1(B,b)
  \]
  Such an $X$ exists by the fundamental theorem.
  
\end{prop}
\begin{proof}
  
  \textit{$(1 \iff 2)$}
  $\pi_1(B,b)$ is a transitive $\pi_1(B,b)$-set 
  if and only if $X$ is path connected.
  For any $x \in \FIB_X(b)$,
  $\pi_1(X,x) \iso \stab(x)$.
  The result follows.

\end{proof}

\end{document}