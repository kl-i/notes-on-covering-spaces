\documentclass[./main.tex]{subfiles}
\begin{document}

\begin{dfn}[Lift]
  
  Let $B \in \TOP$, $X \to B$, $Y \to B$ in $\TOP$.
  Then a \emph{lift} of $Y \to B$ along $X \to B$ is 
  a morphism $f \in \TOP\darrow B(Y,X)$.
\end{dfn}

\begin{prop}[Monodromy Functor]
  
  Let $B \in \TOP$ be locally path connected and $X \in \COV(B)$.
  (It follows that $X$ is locally path connected.)
  Then the \emph{monodromy functor of $X$} is defined as the $\Pi_1(B)$-set : 
  \begin{align*}
    \FIB_X : \Pi_1(B) &\to \SET \\
    b &\mapsto \downarrow\inv b \\
    [\ga] &\mapsto \FIB_X([\ga]) : 
      \downarrow\inv s(\ga) \to \downarrow\inv t(\ga), 
      x \mapsto \ga_x(1)
  \end{align*}
  where $\ga_x$ is any lift of any representative of $[\ga]$,
  such that $\ga_x(0) = x$.
  This gives rise to a functor : 
  \[
    \FIB : \COV(B) \to \SET^{\Pi_1 B}
  \]
  Furthermore, for each $X \in \COV(B)$, 
  we can recover the fundamental groupoid of $X$ via 
  \[
    \int_{\Pi_1(B)} \FIB_X \simeq \Pi_1(X) 
  \]
  where the former is the category of elements of $\FIB_X$.
  Hence \begin{itemize}
    \item $X$ is path-connected if and only if 
    $\FIB_X$ is transitive, 
    i.e. for all $(b,x),(b_1,x_1) \in \int_{\Pi_1(B)} \FIB_X$,
    there exists a morphism $[\ga]$ of $\Pi_1(B)$ such that 
    $\FIB_X([\ga])(x) = x_1$.
    \item for all $(b,x) \in \int_{\Pi_1(B)} \FIB_X$,
    the induced group morphism $\pi_1(X,x) \to \pi_1(B,b)$
    maps $\pi_1(X,x)$ isomorphically to $\stab(x)$.
  \end{itemize}
\end{prop}
\begin{proof}
  
  To define the $\Pi_1(B)$-action on fibres,
  we need to be able to lift paths uniquely \emph{and}
  for path-homotopies to lift. 
  This is exactly what the local trivisalisations of coverings 
  allow us to achieve.
  We first prove paths lift uniquely. 
  \begin{lem}[Unique Path Lifting]
    Let $B \in \TOP$, $X \in \COV(B)$.
    Then $X$ satisfies \emph{unique path lifting},
    meaning for all commuting squares of the form : 
    \begin{cd}
      \bullet \ar[d,"0"{swap}] \ar[r]
        & X \ar[d] \\
      I \ar[r] \ar[ru,dashed,"!"]
        & B
    \end{cd}
    there exists a unique morphism in the diagonal 
    such that the whole diagram commutes. 
    Such a diagonal morphism is called a \emph{lift}
    of the morphism $I \to B$.
    \begin{proof1}
      Let $\ga : I \to B$ and $x \in X$ in the fibre over $\ga(0)$.
      Then there exists a set $\UU$ consisting of opens of $B$
      trivialising $X$ such that $\ga I \subs \bigcup \UU$. 

      The idea is that each $U \in \UU$ allows us to lift a part of $\ga I$
      and compactness of $I$ allows for induction. 
      Since $I$ is compact, there exists a partition 
      $\set{0 = t_0 < \cdots < t_n = 1}$ of $I$
      such that for each $t_i < t_n$, 
      $\ga [t_i,t_{i+1}]$ is in some $U_i \in \UU$.
      Suppose by induction we have a unique lift 
      $\bar{\ga_{n-1}} : [0,t_{n-1}] \to X$ of $\ga : [0,t_{n-1}] \to B$.
      Let $U_n \in \UU$ with $\ga [t_{n-1},t_n] \subs U_n$.
      Let $s_n : U_n \to X$ be a section such that 
      $\bar{\ga_{n-1}}(t_{n-1}) \in s_n U_n$. 
      The define a lift $\bar{\ga} : I \to X$ by patching together 
      $\bar{\ga_{n-1}}$ and $s_n \circ \res{\ga}{[t_{n-1},t_n]}$. 
      This lift is unique because 
      any lift $\tilde{\ga} : I \to X$ of $\ga$
      must restrict to a lift of $\ga : [0,t_{n-1}] \to B$,
      and thus $\res{\tilde{\ga}}{[0,t_{n-1}]} = \bar{\ga_{n-1}}$
      by uniqueness of $\bar{\ga_{n-1}}$ and 
      finally $\tilde{\ga}$ must also agree with $\bar{\ga}$ on 
      $[t_{n-1},t_n]$ since $\tilde{\ga}[t_{n-1},t_n] \subs s_n U_n$ 
      and $s_n$ is a homeomorphism onto its image. 
    \end{proof1}
  \end{lem}
  For lifting homotopies, we prove a more general lemma : 
  \begin{lem}[Unique Homotopy Lifting]
    Let $B \in \TOP$ be locally path connected and $X \in \COV(B)$.
    (It follows that $X$ is also locally path connected.)
    Then $X$ satisfies \emph{unique homotopy lifting with respect to 
    locally connected spaces}, 
    meaning for all commuting squares 
    \begin{cd}
      Y \ar[d,"\id{Y} \times 0"{swap}] \ar[r]
        & X \ar[d] \\
      Y \times I \ar[r] \ar[ru,dashed,"!"]
        & B
    \end{cd}
    where $Y$ is locally connected, 
    there exists a unique morphism in the diagonal 
    such that the diagram commutes. 
    Such a morphism is called a \emph{lift} of 
    the morphism $Y \times I \to B$.
    \begin{proof1}

      \textit{(Existence)}
      \textit{(Local Lifts)}
      Let $y \in Y$.
      We show the existence of an open neighbourhood $U_y$ of $y$
      with a lift $\bar{H}_y : V_y \times I \to X$ 
      of $H : U_y \times I \to B$.
      Let $\UU$ be an open cover of $B$ trivialising $X$.
      Since $B$ is locally path connected, 
      we can WLOG assume $\UU$ consists of path connected opens. 
      Now, for every $t \in I$, 
      there exists $\ep_t > 0$ and $V_t$ open neighbourhood of $y$ such that 
      $H V_t \times [t-\ep_t,t+\ep_t] \subs U_t$ for some $U_t \in \UU$.
      By compactness of $I$, 
      there exists a partition $\set{0 = t_0 < \dots < t_n = 1}$ of $I$
      and open neighbourhoods $(V_i)$ of $y$ such that 
      $H V_i \times [t_i,t_{i+1}] \subs U_i$ for some $U_i \in \UU$.
      We can now take $V_y = \bigcap_i V_i$ as 
      a single open neighbourhood of $y$ such that 
      $H V_y \times [t_i,t_{i+1}] \subs U_i$ for some $U_i \in \UU$.
      We construct a lift $\bar{H}_y$ inductively.
      By local connectedness of $Y$,
      we can WLOG assume $V_y$ is connected, which we will use. 
      Suppose by induction we have a lift 
      $\tilde{H} : V_y \times [0,t_{n-1}] \to X$
      of $H : V_y \times [0,t_{n-1}] \to X$. 
      Let $U_n \in \UU$ that covers $H V_y \times [t_{n-1},t_n]$.
      By connectedness of $V_y$,
      $\tilde{H} V_y \times t_{n-1}$ lies within 
      the image of a section $s_n : U_n \to X$. 
      Hence we can define a lift $\bar{H} : V_y \times I \to X$
      by patching together $\tilde{H}$ and 
      $s_n \circ \res{H}{V_y \times [t_{n-1},t_n]}$.
      
      \textit{(Global Lift)}
      We have an open cover $\YY$ of $Y$ 
      and for each $V \in \YY$ a lift $\bar{H}_V: V \times I \to X$
      of $H : V \times I \to B$.
      For any two $V, W \in \YY$,
      $\bar{H}_V$ and $\bar{H}_W$ both restrict to 
      lifts of $V \times I \cap W \times I = (V \cap W)\times I$. 
      But these give lifts of paths starting in $V \cap W$ and 
      lifts of paths are unique by the previous lemma,
      so $\bar{H}_V$ and $\bar{H}_W$ agree on $V\times I \cap W \times I$.
      Thus these lifts patch togther to give a global lift 
      $\bar{H} : Y \times I \to X$ of $H$.

      \textit{(Uniqueness)}
      Let $\bar{H}, \bar{H}_1 : Y \times I \to X$ be 
      lifts of $H : Y \times I \to B$. 
      Again, these restrict to lifts of paths ${y} \times I \to B$,
      which are unique by the previous lemma 
      so $\bar{H} = \bar{H}_1$.
      
    \end{proof1}
  \end{lem}
  We can now define the $\Pi_1(B)$-action.
  Let $[\ga]$ be a morphism in $\Pi_1(B)$.
  Choose a representative $\ga$.
  Define \[
    \FIB_X([\ga]) : \darrow\inv s([\ga]) \to \darrow\inv t([\ga]) 
    := x \mapsto \ga_x(1)
  \]
  where $\ga_x : I \to X$ is the unique lift of $\ga$ with $\ga_x(0) = x$.
  We now need to show this is independent of the choice of $\ga$.
  Let $\ga^1$ be another representative of $[\ga]$.
  So we have a homotopy $H : I \times I \to B$ from  
  $\ga$ to $\ga^1$ that fixes endpoints. 
  Since $I$ is locally connected, we have a lift : 
  \begin{cd}
    I \ar[d,"\id{Y} \times 0"{swap}] \ar[r,"\ga_x"]
      & X \ar[d] \\
    I \times I \ar[r,"H"] \ar[ru,dashed,"\bar{H}"]
      & B
  \end{cd}
  We hope that $\bar{H}$ gives a path-homotopy.
  Well, \begin{enumerate}
    \item Restricted to $0 \times I$, 
    $\bar{H}$ gives a lift of the constant point $s([\ga])$.
    By uniqueness of path lifting, $\bar{H}$ must be 
    constant along $0 \times I$.
    Similarly, $\bar{H}$ is the constant point $t([\ga])$
    along $1 \times I$.
    \item Now restricted to $I \times 1$, 
    $\bar{H}$ gives a lift of $\ga^1$ starting at $s([\ga])$.
    By uniquenss of path lifting, 
    $\bar{H}$ must be $\ga^1_x$ along $I \times 1$.
  \end{enumerate}
  Hence, $\bar{H}$ is indeed a homotopy from $\ga_x$ to $\ga^1_x$
  fixing end points,
  i.e. $[\ga_x] = [\ga^1_x]$.
  In particular, $\ga_x(1) = \ga^1_X(1)$ so 
  $\FIB_X([\ga])(x)$ is well-defined. 

  \textit{($\FIB$)}
  Let $f \in \COV(Y,X)$.
  Then indeed for every morphism $[\ga]$ in $\Pi_1(B)$,
  we have \begin{cd}
    \FIB_X(s(\sqbrkt{\ga})) \ar[d,"\FIB_X(\sqbrkt{\ga})"] \ar[r,"f"]
      & \FIB_Y(s(\sqbrkt{\ga})) \ar[d,"\FIB_Y(\sqbrkt{\ga})"] \\
    \FIB_X(t(\sqbrkt{\ga})) \ar[r,"f"]
      & \FIB_Y(t(\sqbrkt{\ga})) \\
  \end{cd}
  since for every $x$ in the fibre over the source of $[\ga]$
  and any lift $\ga_x$ of $[\ga]$ starting at $x$,
  $f \circ \ga_x$ is a lift of $[\ga]$ starting at $f(x)$.

  \textit{(Furthermore)}
  We describe the functor $\int_{\Pi_1(B)} \FIB_X \to \Pi_1(X)$ :
  \begin{itemize}
    \item for each object $(b,x)$, map it to $x$.
    \item for each morphism $[\ga] \in \int_{\Pi_1(B)} \FIB_X ((b,x),(b_1,x_1))$,
    map it to $[\ga_x]$ where 
    $\ga_x$ is any lift of $\ga$ starting at $x$.
    We have seen this is well-defined and 
    by the assumption of $\FIB_X([\ga])(x) = x_1$,
    $[\ga_x] \in \Pi_1(X)(x,x_1)$ indeed.
    \item Functoriality follows from uniqueness of path liftings.
  \end{itemize}
  The functor is clearly essentially surjective.
  Faithfulness comes from projecting paths 
  back down $\Pi_1(X) \to \Pi_1(B)$.
  We have seen fullness. 

  \textit{(Hence)} Straightforward. 
\end{proof}

\begin{rmk}
  The monodromy functor gives one side of the equivalence
  $\COV(B) \simeq \SET^{\Pi_1 B}$.
  For the quasi-inverse functor to exist,
  the extra condition on $B$ is the following,
  which says that ``there are enough opens $U$ of $B$ which are 
  determined by their fundamental groupoids''.
\end{rmk}

\begin{lem}[Characterisation of Semi-Locally Simply Connected]
  
  Let $B \in \TOP$ locally path connected.
  Given an open $U$ of $B$ and $b \in U$, TFAE : 
  \begin{enumerate}
    \item $U$ is path connected and 
    the obvious morphism $\pi_1(U,b) \to \pi_1(B,b)$ is trivial.
    \item The following being a bijection : 
    \[
      b\darrow \Pi_1U \to U
    \] 
  \end{enumerate}
  When $U$ and $b$ satisfies any (and thus both) of the above,
  call $b$ a \emph{centre} of $U$.
  If $U$ satisfies the above \emph{for some} $b \in U$,
  then call $U$ a \emph{centred} open.\footnote{
    I made up this terminology to avoid 
    repeating long phrases in the proof.
  }
  Then TFAE : 
  \begin{enumerate}
    \item There exists an open cover $\UU$ of $B$ consisting of 
    $U$ such that every $b \in U$ is a centre of $U$.
    \item For every $b \in B$, 
    there is a neighbourhood base of opens $U$ 
    with $b$ as a centre. 
    \item There exists a cover $\UU$ of $B$ consisting of 
    centred opens. 
  \end{enumerate}
  We say $B$ is \emph{semi-locally simply connected} 
  when it satisfies any (and thus all) of the above. 
\end{lem}
\begin{proof}

  \textit{($1 \iff 2$ for $U$ and $b$)}
  $U$ being path connected corresponds to 
  $b\darrow \Pi_1U \to U$ being surjective.
  It suffices to prove $\pi_1(U,b) \to \pi_1(B,b)$ trivial 
  if and only if $b\darrow \Pi_1U \to U$ injective.
  Forwards, given two morphisms $[\ga],[\ga_1]$ in $\Pi_1 U$ 
  with source at $b$ and same target,
  $[\ga]\inv [\ga_1] \in \pi_1(U,b)$.
  Triviality of $\pi_1(U,b) \to \pi_1(B,b)$ implies 
  $[\ga] = [\ga_1]$ as morphisms in $\Pi_1 B$, in particular in $\Pi_1 U$.
  The converse is easy. 
  
  Now for equivalent conditions of $B$ semi-locally simply connected.
  $(1 \implies 2)$ Use local path connectedness of $B$ and functoriality 
  of $\pi_1(-,b)$.
  $(2 \implies 3)$ Obvious. 
  $(3 \implies 1)$
  Let $\UU$ be an open cover of $B$ such that 
  for all $U \in \UU$, there exists a centre $b$ of $U$.
  It suffices to show for other $b_1 \in U$,
  $\pi_1(U,b_1) \to \pi_1(B,b_1)$ is also trivial. 
  By assumption, there exists $[\ga] \in \Pi_1U(b,b_1)$,
  so we have a commutative square
  \begin{cd}
    \pi_1(U,b) \ar[d,"\iso"{swap}] \ar[r,"1"]
      & \pi_1(X,b) \ar[d,"\iso"] \\
    \pi_1(U,b_1) \ar[r]
      & \pi_1(X,b_1) \\
  \end{cd}
  where the vertical maps are ``conjugation'' by $[\ga]$.
  This proves the bottom horizontal morphism is trivial. 

\end{proof}

\begin{prop}[Fundamental Theorem of Covering Spaces
  \footnote{As called on nLab page on covering space.}]
  
  Let $B \in \TOP$ be locally path connected and 
  semi-locally simply connected.
  Consider the functor $\int_{\Pi_1 B} : \SET^{\Pi_1 B} \to \SET \darrow B$
  that sends $X$ to its category of elements $\int_{\Pi_1 B} X$,
  which we then view as a set with a set-morphism down to $B$.
  Then \begin{enumerate}
    \item we can promote the $\int_{\Pi_1 B}$ to a functor 
    $\SET^{\Pi_1 B} \to \COV B$. 
    \item $\int_{\Pi_1 B}$ and $\FIB$ form an 
    equivalence of categories \[
      \COV(B) \simeq \SET^{\Pi_1 B}
    \]
  \end{enumerate}
\end{prop}
\begin{proof}
  \textit{(1)}
  Let $\rho \in \SET^{\Pi_1 B}$.
  By assumption, $B$ has a topological base of centred opens. 
  These will give the topology on $\int_{\Pi_1 B} \rho$ 
  as well as the trivialising opens of the projection. 

  \textit{(1)(Topology)}
  For $(b,x) \in  \int_{\Pi_1 B} \rho$ and 
  an open $U$ of $B$ centred at $b$,
  define \[
    U[b,x] := \set{\rho([\ga](x) \st [\ga] \in b \darrow \Pi_1 U}
  \]
  Then the set of these give a topological base 
  because centred opens form a topological base for $B$.
  We topologise $\int_{\Pi_1 B} \rho$ using this topological base.

  \textit{(1)(Projection Topological)}
  To show $\int_{\Pi_1 B} \rho \to B$ is a morphism of topological spaces,
  since $B$ has centred opens as a topological base,
  it suffices for each centred $U$, say at some $b$,
  to have preimage \[
    \darrow\inv U = \bigsqcup_{x \in \rho(b)} U[b,x]
  \]
  The equality is straight forward from $U$ being centred at $b$ and 
  disjoint union follows from $\Pi_1 U$ being a groupoid. 

  \textit{(1)(Covering)}
  We saw already that $\darrow\inv U = \bigsqcup_{x \in \rho(b)} U[b,x]$
  as sets. 
  Since $\int_{\Pi_1 B} \rho$ has topology generated by 
  the $V[b,x]$'s, 
  to show $U[b,x] \to U$ is a topological isomorphism,
  it suffices to show for all $U$ with centres $b$ and $x \in \rho(b)$,
  $U[b,x]$ maps to $U$ bijectively. 
  And it does because $U$ is centred!
  
  \textit{(1)(Functorial)}
  Let $f \in \SET^{\Pi_1 B}(\rho,\rho_1)$.
  We need to show the induced 
  $f_* : \int_{\Pi_1 B} \rho \to \int_{\Pi_1 B} \rho_1$
  is a topological morphism over $B$.
  Suffices for basic opens $U[b,y]$ of $\int_{\Pi_1 B} \rho_1$
  to have preimage \[
    \darrow\inv U[b,y] = \bigcup_{x \in f_b\inv(y)} U[b,x]
  \]
  where $f_b : \rho(b) \to \rho_1(b)$.
  $\sups$ is clear. 
  Let $(b_1,x_1)$ be in the preimage of $U[b,y]$.
  By definition, there exists $[\ga] \in \Pi_1 U(b,b_1)$ with 
  $\rho[\ga](y) = f_{b_1}(x_1)$.
  Then by naturality of $f$, we have 
  \begin{cd}
    \rho(b) \ar[d,"\rho\sqbrkt{\ga}"{swap}] \ar[r,"f_b"]
      & \rho_1(b) \ar[d,"\rho_1\sqbrkt{\ga}"]\\
    \rho(b_1) \ar[r,"f_{b_1}"]
      & \rho_1(b_1)
  \end{cd}
  Since $\Pi_1 B$ is a groupoid, $\rho[\ga]$ are isomorphisms of sets,
  so there exists $x \in \rho(b)$ with $\rho[\ga](x) = x_1$
  and $f_b(x) = y$,
  i.e. we have the other inclusion. 

  \textit{(2)($\FIB \circ \int_{\Pi_1 B} \iso \id{}$)}
  We know that $\FIB\brkt{\int_{\Pi_1 B} \rho}(b) = \rho(b)$
  for $b \in \Pi_1 B$.
  So it suffices that for all morphisms $[\ga]$ in $\Pi_1 B$,
  \[ 
    \FIB_{\int_{\Pi_1 B} \rho} [\ga] = \rho[\ga]
  \]
  which will actually show $\FIB \circ \int_{\Pi_1 B} = \id{}$.
  The left is topological and right is algebraic. 
  We bridge from left to right by using compactness of paths 
  to break $[\ga]$ into finitely many pieces that 
  lie within basic opens, where things are algebraic. 

  Let $[\ga]$ be a morphism in $\Pi_1 B$. 
  Since $B$ is semi-locally simply connected, 
  we have an open cover $\UU$ of $B$ such that 
  every $b \in U$ is a centre of $U$.
  By compactness of $[0,1]$,
  there exists morphisms $[\ga_0],\dots,[\ga_n]$ in $\Pi_1 B$
  such that $[\ga] = [\ga_n] \circ \cdots \circ [\ga_0]$ and 
  $[\ga_i]$ is $\Pi_1 U_i$ for some $U_i \in \UU$.
  Let $x \in \rho(s[\ga]) = \rho(s[\ga_0])$.
  By unique path lifting, 
  \[ 
    \brkt{\FIB_{\int_{\Pi_1 B} \rho}[\ga_0]} (x) \in 
    U_0\sqbrkt{s[\ga_0],x} \cap \rho(t[\ga_0])
    = U_0\sqbrkt{t[\ga_0],\rho[\ga_0](x)} \cap \rho(t[\ga_0])
    = \set{\rho[\ga_0](x)}
  \]
  Thus $\brkt{\FIB_{\int_{\Pi_1 B} \rho}} [\ga_0](x)
  = \rho[\ga_0](x)$.
  The same goes for all $i$, giving 
  \begin{align*}
    \FIB_{\int_{\Pi_1 B} \rho}[\ga]
    = \FIB_{\int_{\Pi_1 B} \rho}[\ga_n] \circ \cdots \circ 
    \FIB_{\int_{\Pi_1 B} \rho}[\ga_0]
    = \rho[\ga_n] \circ \cdots \circ \rho[\ga_0]
    = \rho[\ga]
  \end{align*}

  \textit{(2)($\int_{\Pi_1 B} \circ \FIB \iso \id{}$)}
  There's an obvious set isomorphism : 
  \[
    \int_{\Pi_1 B} \FIB_X \to X
  \]
  We have a topological base of $B$ consisting of centred opens,
  which \emph{also} trivialise $X$,
  so $X$ also has a topological base consisting of centred opens,
  which map isomorphically down to centred opens of $B$.
  Let us call such opens of $X$ basic for duration of the rest of the proof.
  It suffices for preimage of basic opens of $X$ to be open. 
  Let $U$ be a basic open of $X$ and $x$ a centre of $U$.
  Let $b \in B$ be the projection of $x$ and $U_0$ the projection of $U$.
  The claim is that the preimage of $U$ is $U_0[b,x]$. 
  This is clear since 
  \begin{cd}
    U_0\sqbrkt{b,x} \ar[r] \ar[rd,"\iso"{swap}]
      & U \ar[d,"\iso"{swap}] \\
      & U_0
  \end{cd}
\end{proof}

\end{document}