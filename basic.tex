\documentclass[./main.tex]{subfiles}
\begin{document}

\begin{dfn}[Fundamental Groupoid, Fundamental Group]

  Let $B \in \TOP$.
  Define the \emph{fundamental groupoid of $B$}, $\Pi_1(B)$, to be 
  following category : 
  \begin{itemize}
    \item $\Pi_1(B)$ has $B$ as its collection objects.
    \item For $b, b_1 \in B$,
    $\Pi_1(B)(b,b_1)$ consists of paths from $b$ to $b_1$ 
    up to path-homotopy. 
    \item Composition is concatenation of paths up-to-homotopy.
    That this is well-defined and associative is an unenlightening exercise.
    \item For $b \in B$, $\id{b}$ is the homotopy class of the 
    constant path at $b$.
  \end{itemize}
  All morphisms of $\Pi_1 B$ are isomorphisms,
  i.e. $\Pi_1 B$ is indeed a groupoid.
  This gives a functor 
  \[
    \Pi_1 : \TOP \to \mathbf{Grpd}
  \]
  For $b \in B$, use $\pi_1(B,b)$ to denote the 
  group of automorphisms of $b$ in $\Pi_1(B)$.
  This is called the \emph{fundamental group of $B$ at $b$}.
  This gives a commutative square of functors : 
  \begin{cd}
    \TOP* \ar[r,"\pi_1"] \ar[d]
      & \GRP \ar[d] \\
    \TOP \ar[r,"\Pi_1"]
      & \mathbf{Grpd}
  \end{cd}
  where $\TOP*$ is pointed topological spaces.
\end{dfn}

\begin{dfn}[Covering]

  Let $B \in \TOP$.
  For $X \to B$ and $U \to B$ in $\TOP\darrow B$,
  we say $U$ \emph{trivialises $X$} when 
  there exists discrete $X_U \in \TOP$ satisfying the pullback diagram : 
  \begin{cd}
    X \ar[d]
      & X_U \times U \ar[d] \ar[l] \\
    B 
      & U \ar[l,"\sups"]
  \end{cd}
  where the right vertical morphism is projection in the $U$-component.
  In this case, we call $X_U$ the \emph{generic fibre over $U$}.

  Define the \emph{category of coverings of $B$}, $\COV(B)$, as 
  the full subcategory of $\TOP\darrow B$ consisting of 
  $X \to B$ such that there exists a cover $\UU$ of $B$
  consisting of opens that trivialise $X$.\footnote{
    It is standard to require non-empty generic fibres,
    however I have seen no use of this in the theory so 
    have chosen to not include it. 
  }
  Objects of $\COV(B)$ are called \emph{coverings} of $B$.
  When considering $\COV(B)$,
  $B$ is referred to as the \emph{base space}.
  
\end{dfn}

\begin{rmk}[Big Picture of Covering Spaces]
  
  The invariant we seek to understand is $\Pi_1 B$.
  The actions of a group can tell a lot about the group.
  The main result of covering spaces is that 
  given sufficiently nice base space $B$,
  covering spaces tell us everything about actions of the fundamental groupoid.
  Formally, we have an equivalence of categories : 
  \[
    \COV(B) \simeq \SET^{\Pi_1 B}
  \]
  This is theoretically good,
  but for computations we restrict attention to fundamental groups
  so we can have group theory at our disposal.
  Given a choice of $b \in B$,
  we have a restriction functor : 
  \[
    \SET^{\Pi_1 B} \to \SET^{\pi_1(B,b)}
  \]
  Under additional assumptions on $B$,
  this is an equivalence and will yield : 
  \[
    \COV(B) \simeq \SET^{\pi_1(B,b)}
  \]
  This is called the \emph{Galois theory of covering spaces}.

\end{rmk}
  
\begin{prop}[Permenance Properties]
  
  The following are true : 
  \begin{itemize}
    \item (Base Change) 
    Consider a pullback diagram in $\TOP$ : 
    \begin{cd}
      X \ar[d]
        & Z \ar[d] \ar[l] \\
      B 
        & Y \ar[l]
    \end{cd}
    Then $X \to B$ covering implies $Z \to Y$ covering. 
    The converse is false. 
    \item (Composition)
    Let $X \to Y \to B$ in $\TOP$.
    Then $X \to Y$ and $Y \to B$ coverings
    imply $X \to B$ is a covering. 
    The converse is false. 
    % \item (Partial Converse to Composition)\footnote{
    %   This result is often used in expositions of covering spaces,
    %   however we will not make use of this. 
    % }
    % Let $X \to Y \to B$ in $\TOP$.
    % Assume $Y$ connected.
    % Then $X \to B$, $Y \to B$ coverings imply
    % $X \to Y$ is a covering.
    %
    % Hence,
    % all morphisms between connected coverings over a base
    % are coverings. 
    \end{itemize}
  
\end{prop}
\begin{proof}

  \textit{(Base Change)}
  We have the following commutative cube : 
  \begin{cd}
      & U \times X_U \ar[ld, tail] \ar[dd]
        & 
          & V \times X_U \ar[ll] \ar[ld, tail] \ar[dd]\\
    X \ar[dd]
      & 
        & Z \ar[ll, crossing over] 
          & \\
      & U \ar[ld, tail]
        & 
          & V \ar[ld, tail] \ar[ll] \\
    B 
      & 
        & Y \ar[ll] \ar[uu, leftarrow, crossing over]
          & \\
  \end{cd}
  The faces that are pullback squares are : 
  \begin{itemize}
    \item front, by assumption.
    \item bottom, which gives $V$ an open in $Y$.
    \item left, where $U$ is a trivialising open of $X$ over $B$ and 
    $X_U$ is a discrete space,
    with $U \times X_U \to U$ being projection into first component. 
    \item back, which follows easily. 
    The projection $V \times X_U \to V$ is into the first component.
  \end{itemize}
  The left and back being pullback squares implies 
  $V\times X_U$ is the pullback of 
  $X, V$ over $B$,
  and together with $Z$ being the pullback of $X,Y$ over $B$,
  this implies $V\times X_U$ is the pullback of 
  $Z, V$ over $Y$,
  i.e. isomorphic to the preimage of $V$ under $Z \to Y$.
  This gives $V$ as a trivialising open of $Z$ over $Y$.
  We can hence obtain a trivialising cover for $Z$ over $Y$.

  %Surjectivity of $Z \to Y$ follows easily from that of $X \to B$.
  % It remains to show $Z \to Y$ is surjective. 
  % Let $y : \bullet \to Y$ be a point of $Y$.
  % It follows from this commutative cube :  
  % \begin{cd}
  %   & X\times_B \fall{B}{Y} y \ar[ld, tail] \ar[dd]
  %     & 
  %       & Z\times_Y y \ar[ll] \ar[ld, tail] \ar[dd]\\
  % X \ar[dd]
  %   & 
  %     & Z \ar[ll, crossing over] 
  %       & \\
  %   & \bullet \ar[ld, tail]
  %     & 
  %       & \bullet \ar[ld, tail, "y"] \ar[ll] \\
  % B 
  %   & 
  %     & Y \ar[ll] \ar[uu, leftarrow, crossing over]
  %       & \\
  % \end{cd}

  \textit{(Composition)}
  This is the diagram : 
  \begin{cd}
    X \ar[d]
      & U \times X_U \times Y_U \ar[l,tail] \ar[d] \\
    Y \ar[d]
      & U \times Y_U \ar[d] \ar[l,tail] \\
    B
      & U \ar[l,tail]
  \end{cd}
  The bottom square is 
  the pullback square of $X$ over $B$ along 
  a trivialising open $U$ of $Y$ over $B$.
  We can then shrink $U$ such that 
  each $U \to U \times X_U \to Y$ is (isomorphic to) a trivializing open
  of $X$ over $Y$,
  hence the top square. 
  Since $X_U, Y_U$ are discrete, $X_U \times Y_U$ is discrete as well.
  The ``rectangle'' is a pullback diagram because 
  the inner two are.
  Thus $U$ is a trivialising open of $X$ over $B$.
  This gives an open cover of $B$ trivialising $X$.

  %Surjectivity is clear.

  % \textit{(Partial Converse to Composition)}
  % A common refinement of two open covers of $B$ trivialising $X,Y$ respectively
  % will trivialise both $X,Y$.
  % So let $\UU$ be an open cover of $B$ that trivialises both $X, Y$.
  % For $U \in \UU$, we have 
  % \begin{cd}
  %   X \ar[d]
  %     & U \times X_U \ar[l,tail] \ar[d] \\
  %   Y \ar[d]
  %     & U \times Y_U \ar[d] \ar[l,tail] \\
  %   B
  %     & U \ar[l,tail]
  % \end{cd}
  % where the bottom square and whole rectangle are pullback squares,
  % and $X_U, Y_U$ discrete spaces.
  % The above square is hence also a pullback square.
  % Then the opens $U \to U \times Y_U \to Y$
  % ranging over $U \in \UU$ give an open cover of $Y$
  % trivialising $X$.
  %
  % Note that we have not used connectedness of $Y$ so far.
  % We do to prove $X \to Y$ is surjective.
  % Since $X \to Y$ is an open map,
  % its image is open,
  % so to show surjectivity it suffices $Y\minus \fall{}{} X$ be open
  % by connectedness of $Y$.
  % Let $y \in Y\minus\fall{}{} X$.
  % There exists an open neighbourhood $U$ of $y$ 
  % that trivialises $X$. 
  % Since the fibre over $y$ is empty,
  % it follows that $U \subs Y\minus \fall{}{} X$.

\end{proof}

\end{document}