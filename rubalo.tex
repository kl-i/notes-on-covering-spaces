\documentclass[./main.tex]{subfiles}
\begin{document}
  


\begin{prop}[Galois Covers]
Let $X \to B$ be a covering. 
For any group $G \subs \AUT_B X$, TFAE : 
\begin{enumerate}
  \item The canonical morphism from the quotient to $B$ is an isomorphism : 
  \begin{cd}
    X \ar[d] \ar[r]
      & X / G \ar[ld,"\iso"]\\
    B 
      &
  \end{cd}
  \item For all $b \in B$, 
  $G$ acts transitively on the fiber over $b$.
\end{enumerate}
The cover $X \to B$ is called \emph{Galois}
when there exists a group $G \subs \AUT_B X$
satisfying any (and thus both) of $(1), (2)$. 
\footnote{
  These are often also called \emph{normal covers}
}
Furthermore, if $X$ is connected,
then any such $G = \AUT_B X$ and 
the fibres are in fact $G$-torsors.\footnote{
  One also says $X \to B$ is a \emph{principal $G$-bundle over $B$}.
}

\end{prop}
\begin{proof}
  $(1 \implies 2)$ ok.
  $(2 \implies 1)$ $X \to B$ is an open map and the assumption 
  is that the fiber over $b$ is exactly the orbit of $b$.

  \textit{(Furthermore)}
  We prove the following lemma : 
  \begin{lem}[Amazing]
    Let $X, Y \in \COV(B)$ of some $B \in \TOP$.
    Suppose $\al, \be \in \COV(B)(X,Y)$ that agree on at least one point
    and $X$ is connected.
    Then $\al = \be$.
    \begin{proof1}
      Consider the pullback diagram :  
      \begin{cd}
        X \ar[d,"\al\times\be"{swap}]
          & U \ar[l,"\sups"{swap}] \ar[d]\\
        Y \times_B Y 
          & Y \ar[l,"\De"]
      \end{cd}
      It suffices $U$ is all of $X$.
      By assumption, $U \neq \nothing$
      so connectedness of $X$ implies it suffices $\De(Y)$ is clopen.
      Well, $Y \to B$ local homeomorphism implies 
      $Y \to B$ is an open map, which implies $\De(Y)$ is open.
      On the other hand, 
      $Y \to B$ is separated so $\De(Y)$ is closed.
    \end{proof1}
  \end{lem}
  
  Now let $\si \in \AUT_B X$.
  By the lemma, it suffices to give $x_0 \in X$ and $\si_0 \in G$ such that 
  $\si(x_0) = \si_0(x_0)$.
  By assumption, $G$ acts transitively on the fiber that $x_0$ is in
  and $\si(x_0)$ is in the fibre so we gucci. 
  A $G$-torsor is a non-empty, free and transitive $G$ action. 
  The fibres are non-empty and transitive $G$-actions.
  Freeness comes from the lemma. 

\end{proof}

\begin{lem}[Sufficient Condition for Galois Covers]

  Let $X \in \TOP$, $G$ subgroup of $\AUT_\TOP X$.
  Consider the propositions : 
  \begin{enumerate}
    \item $X \to X/G$ is a covering.
    \item $G$ acts on $X$ \emph{evenly},
    meaning for all $x \in X$, 
    there exists an open neighbourhood $U^x$ of $x$ such that 
    for all $\si,\si_1 \in G$, 
    $\si \neq \si_1$ implies $\si U^x \cap \si_1 U^x = \nothing$.
  \end{enumerate}
  Then we have :
  \begin{itemize}
    \item $1 \limplies 2$
    \item If $X$ connected and $X/G$ is locally connected,
    $1 \implies 2$.
    \item There exists $X$ disconnected with $1 \not\implies 2$.
  \end{itemize}
\end{lem}
\begin{proof1}

  $(1 \limplies 2)$
  Let $x \in X$ and $U^x$ an open neighbourhood where 
  ``$U^x$ separates the orbit of $x$''.
  It suffices to show the preimage of the image of $U^x$ is 
  $\bigcup_{\si \in G} \si U^x$.
  This is clear.

  $(1 \implies 2)$
  Let $X$ connected and locally connected.
  Let $x \in X$.
  By assumption, there exists an open $U$ of $X/G$ such that 
  we have the pullback diagram : 
  \begin{cd}
    X \ar[d]
      & U \times \orb(x) \ar[d] \ar[l] \\
    X/G 
      & U \ar[l,"\sups"]
  \end{cd}
  The intuition is that
  ``$G$ acts on the slices of $U$ in $U \times \orb(x)$ according 
  its action on $\orb(x)$''.
  We choose $U^x := $ image of $U \times \set{x}$.
  Since $\orb(x) \iso G / \stab(x)$ as $G$-actions,
  and $\stab(x)$ is trivial by connectedness of $X$,
  we use locally connectedness of $X$ to show that 
  for $\si \neq \id{X}$, 
  we effectively have $\si U^x \cap U^x = \nothing$.

  Well, $X/G$ is locally connected, so by shrinking $U$ we can 
  WLOG assume $U$ is connected. 
  Then for $\si \in G \minus \stab(x)$,
  $\si(x) \notin U^x$ implies $\si U^x \cap U^x = \nothing$ 
  since $U^x \iso U$ is connected.

  $(1 \not\implies 2)$
  Consider the $\Z/2\Z$-action \[
    a \,\,\,\,\,\,\,\,\,\,\,\,\, b \longleftrightarrow c
  \]
  where $\set{a,b,c}$ is given discrete topology.
  This satisfies $(1)$ but $a$ breaks $(2)$.
  
\end{proof1}

\begin{prop}[``Fundamental Theorem of Galois Theory'']
  
  Let $\CCOV(B)$ be the category of connected coverings 
  of a \emph{locally connected} $B \in \TOP$.
  Let $X \in \CCOV(B)$ be Galois and $G := \AUT_B X$.
  Consider $X \darrow \CCOV(B)$ and $\SUB\GRP(G)$.
  Then we have an equivalence of categories : \footnote{
    By considering $(X\darrow \CCOV(B))/\iso$ instead,
    one obtains an isomorphism of categories.
    Having to pick representatives many times in a proof 
    feels dirty though. 
  }
  \begin{align*}
    X\darrow\CCOV(B) &\to \SUB\GRP(G) \\
    (X \to Y) &\mapsto \AUT_Y X
  \end{align*}
  Furthermore, for $X \to Y$ in $X\darrow \CCOV(B)$,
  $Y \to B$ is Galois if and only if 
  $\AUT_Y X$ is a normal subgroup of $\AUT_B X$.
  In this case, we have the SES of groups : 
  \[
    1 \to \AUT_Y X \to \AUT_B X \to \AUT_B Y \to 1
  \]
\end{prop}
\begin{proof}
  
  \textit{(Essentially Surjective)}
  Let $H \in \SUB\GRP(G)$.
  If $X \to X/H$ is a covering, 
  then it is Galois : 
  for $x_0 \in X$ an $\si_0 \in \AUT_{X/H} X$,
  since $X \to B$ is Galois,
  there exists $\si \in G$ such that $\si(x_0) = \si_0(x_0)$,
  and hence $\si = \si_0 \in H$ by connectedness of $X$.
  In particular, $H = \AUT_{X/H} X$.

  So it suffices to prove $X \to X/H$ is a covering.
  Since $X$ is connected and $B$ is locally connected,
  the action of $G$ on $X$ is even.
  This implies the action of $H$ on $X$ is also even 
  and hence $X \to X/H$ is a covering. 
  
  \textit{(Faithful)}
  Any $X \to Y$ in $X\darrow\CCOV(B)$ is a covering,
  in particular a surjection and hence an epimorphism in $\CCOV(B)$.

  \textit{(Full)}
  Let $X \to Y$, $X \to Z$ in $X\darrow\CCOV(Z)$ such that 
  $\AUT_Y X \subs \AUT_Z X$.
  Then we have 
  \begin{cd}
      & Y \\
    X \ar[ru] \ar[r] \ar[rd]
      & X/\AUT_Y X \ar[u,dashed] \ar[d,dashed]\\
      & Z
  \end{cd}
  It thus suffice that $X/\AUT_Y X \to Y$ is an isomorphism,
  i.e. $X \to Y$ is Galois.
  Let $x_0, x \in X$ be in the same fibre over $Y$
  ($X$ non-empty because the empty case is trivial).
  Then since $X \to B$ is Galois,
  there exists $\si \in G$ such that $\si(x_0) = x$.
  Now $\fall{}{Y} \circ \si$ and $\fall{}{Y}$ agree on $x_0$
  and since $X$ is connected, 
  we have $\si \in \AUT_Y X$.

  \textit{(Furthermore)}
  Let $X \to Y \in X\darrow\CCOV(B)$.
  
  Assuming $Y \to B$ Galois, 
  we construct a group morphism $\AUT_B X \to \AUT_B Y$ 
  with kernel $\AUT_Y X$.
  \begin{lem}
    Let $Y \to B$ be a connected Galois covering.
  
    Then for all connected coverings $X \to B$, 
    $G$ acts freely and transitively on $\COV(B)(X,Y)$.\footnote{
      I have a feeling this should also imply $Y \to B$ Galois,
      but haven't proved it nor found a counter example. 
    }
    \begin{proof1}
    Freeness follows from connectedness of $Y$ and the amazing lemma.
    Now for transitive, let $\ph, \psi \in \COV(B)(X,Y)$.
    Let $b \in B$ and $y \in Y$ in the fibre over it. 
    (If $B = \nothing$, all is well.)
    Then $\ph(y), \psi(y)$ are in the fibre over $b$,
    so there exists $\si \in \AUT_B Y$ such that 
    $\si(\ph(y)) = \psi(y)$.
    It follows from connectedness of $Y$ and the amazing lemma 
    that $\si \circ \ph = \psi$.
    \end{proof1}
  \end{lem}
  The lemma gives a well-defined group morphism $\AUT_B X \to \AUT_B Y$,
  which indeed has kernel $\AUT_Y X$.
  
  Now assume $\AUT_Y X$ is normal in $\AUT_B X$.
  The following lemma is the essence : 
  \begin{lem}[3rd Isomorphism theorem for Actions]
    
    Let $X \in G\SET$ for a group $G$, $H \in \SUB\GRP(G)$ normal.
    Then $X/H$ inherits an action from $G/H$ and 
    $(X/H)/(G/H)$ has the UP of $X/G$.
    \begin{proof1}
      For any $\si \in G$,
      we have : 
      \begin{cd}
        X \ar[r,"\si"] \ar[d] & X \ar[d] \\
        X/H \ar[r,"\bar{\si}"] & X/H
      \end{cd}
      where $\bar{\si}$ comes from the UP of $X/H$
      since $\si H = H \si$ implies $X \to X \to X/H$ 
      preserves the action of $H$.
      This defines a $G$-action and hence a $G/H$-action on $X/H$
      since $H$ acts trivially on $X/H$.

      $(X/H)/(G/H)$ having the UP of $X/G$ is straightforward.
    \end{proof1}
  \end{lem}
  We know that $X \to Y$ is isomorphic to $X \to X/\AUT_Y X$.
  So by the lemma, $Y$ has an action from the quotient $\AUT_B X / \AUT_Y X$
  and the corresponding quotient morphism is $Y \to B$.
  This action is in fact faithful and hence 
  $Y \to B$ is Galois. 

\end{proof}
\end{document}